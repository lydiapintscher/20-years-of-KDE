\chapterwithauthor{Richard J. Moore}{Continuity Through Change}

\authorbio{Richard Moore joined KDE around 1997, first contributing a little widget that displayed an LED. Over time that expanded to cover areas such as scripting engines, task bars, and networking. He is currently the maintainer of the Qt Network module and CTO at Westpoint Ltd, an Internet security company.}

\noindent{}When I first started working on KDE it consisted of a window manager (kwm), a file manager (kfm) and a panel to launch things (kpanel) - it has grown a little since then! It was very easy at the time to make a big contibution to the project simply because there was so much that needed doing. My first application was KPaint, quickly followed by KSnapshot which I wrote mainly so I could make screenshots of KPaint. If you had said to me at the time that KSnapshot would continue to be developed and released for 18 years I wouldn't have believed you.

One of the most surprising things about KDE is how much continuity there has been in the code, and the impact of some early decisions. One major feature was Torben Weis' decision to make the network download facilities of kfm available as a library to other applications. This meant that from the start KDE applications had built-in support for editing files on the internet. This might seem obvious now, when 'The Cloud' is such a big buzz-word and everyone has 24/7 internet access but it was pretty revolutionary at the time. Unlike today, most people weren't on the internet at all (and had often never heard of it). Many of the KDE developers had (by the standards of the time) fast internet access when at university, but at home we had modem access at best. It is only in the last few years with the rise of tools such as One Drive, DropBox and OwnCloud that this kind of facility has become available more widely.

An area where KDE has had a global impact is in the browser market. The original KHTMLW widget was replaced by Lars Knoll with the new standards driven KHTML library. This was later picked up by Apple as the basis of WebKit and now forms the basis of the vast majority of web browsers. I think one of the reasons why KHTML was such a success was due to the way KDE is developed. The use of the LGPL license was obviously a factor, but other rendering engines existed with open licenses. Unlike many other projects, KDE has always had a highly distributed developer base which meant that it was always seen as important to keep APIs clean and understandable, and to make it easy for people to work on the code. The emphasis on clean APIs and clear implementations meant that KHTML was a lot easier to work with than its competitors.

The distributed nature of KDE's development had other benefits too. The first KDE conference organised by Kalle Dalheimer in September 1997 brought together people from all over Europe - it was a very exciting and productive weekend. As well as being a fantastic opportunity to build a community, the amount of code that got written (and rewritten) during the conference was astounding. The KDE internationalisation framework was completed during the meeting, and by the end of the weekend we had the whole desktop working in German ready for the press conference! The meeting also had other highlights such as the first demo of KOffice (written of course by Torben) including the ability to embed different types of document - this would eventually become KParts.

If you look at the modern KDE desktop, there is no doubting that it has evolved from those early days. The level of quality now is unbelievable compared to the early releases. This applies both to the code itself via massive improvements in the testing infrastructure, and much more visibly to the design of the interface itself. A modern KDE desktop is a thing of beauty - the older KDE releases were functional but sadly as engineers we had a big blind spot for graphic design.

KDE as a project has evolved too - most notably it has grown in both numbers and become even more diverse. The growth of a strong KDE community in India to the point where it is now hosting its own KDE conferences is particularly noteworthy. KDE has started taking a more active role in the wider Free Software and Free Culture community for example as an associate organisation of FSFE and through collaboration with groups such as Wikimedia. The continued evolution of the project and community has helped ensure KDE remains relevant despite the ever changing environment it inhabits.

One thing has been constant though throughout my time working with KDE and that has been the people. While the individuals working on the project change over time, the attitude remains the same. The people are intelligent, curious and eager to experiment. I look forward to seeing what they come up with - I'm sure it will be awesome.
