\chapterwithauthor{Aleix Pol i Gonzàlez}{Title} 
\authorbio{Aleix Pol i Gonzàlez is \todo{add bio}}

\todo{add title}

\noindent{}In KDE we are very proud of being a diverse community. We strive to make sure it is, it's not easy. Albeit being an international community it wouldn't take an anthropologist to look at the KDE Community and realize that albeit diverse we aren't equally spread. Instead, we have some defined demographics around age, sex, studies and probably income as well. What I want to discuss today is one of my longstanding focuses since I joined KDE: local communities, or how to offer to my people what we create.

From a creator's perspective, it's useful to turn around and look at who we're dealing with. When it comes to a specific project, one thinks of rather stereotypical people who might be interested in IDE's or spreadsheets, but when we think about what we want to offer as a whole, I can't but think of society in a whole different way. The further away you push it, the clearer it is it's not about adding specific features, but to listen what they need and explain how we can solve the problems they see. To be there when they need assistance and to help them be secure rather than adventurous.

We created KDE España to be able to sustain Akademy-es initially, but over time it has evolved into a platform with a much broader communication spectrum including Akademy-es, a magnificient blog, podcasts, trainings and conferences. This is also important because as soon as you start communicating, people come to you when they feel insecure about how to help. Furthermore, it helps us stay organized and alive.

Interestingly, one of the initiatives we've started from within KDE España has been by narrowing the geographical scope even further. We created a group called Barcelona Free Software, where we offer contents from local FOSS experts and enthusiasts including but not limted to us. This allows people with different sensibilities to those who would attend Akademy or Akademy-es, to come and talk to us. The feedback we reach at such gatherings is interesting as it shows the kind of problems that our people ache from and how they demand for solutions.

In the end, what this experience has reminded me is that I'm here to make sure society can make the best usage of the available technology. Furthermore, in a society increasingly based on information, we need to let it empower the user, the people. We need to remember what's the advantages of offering Free Software solutions beyond technologies, by remembering the end goal is that people have the tools they need to create better contents in every aspect of our lives. And that includes the possibility to adapt the tools we use.

I think one of the most important motivation factors is to see people being able to adopt the solutions you come up with and even more when they use them to create new things. For many of us, it's beyond the comfort zone or even the ideal plan for a weekend but it's worth it to talk to people, it's worth going that local convention and talk about why and how important is to grow our society on Free Software and then gather why that's not the case. Response never ceases to amaze me.
