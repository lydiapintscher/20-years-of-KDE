\chapterwithauthor{Aleix Pol i Gonzàlez}{Local Communities Matter}

\authorbio{Aleix Pol i Gonzàlez has been contributing to KDE since 2007. He started working in software development in the KDE Education area and on KDevelop. He is the president of KDE Espa\~{n}a since 2012 as well as a founding member. Aleix joined the KDE e.V. board of directors in 2014. In his day-job, he is employed by BlueSystems where he has worked with other parts of the community including Plasma and Qt.}

\noindent{}In KDE we are very proud of being a diverse community. We strive to make sure it is - it's not easy. Albeit being an international community it wouldn't take an anthropologist to look at the KDE community and realize that although we are diverse we are not spread out evenly across the globe. Instead, we have some defined demographics around age, sex, studies and probably income as well. What I want to discuss today is one of my longstanding focuses since I joined KDE: local communities, or how to offer to my people what we create.

From a creator's perspective, it is useful to turn around and look at who we are dealing with. When it comes to a specific project, one thinks of rather stereotypical people who might be interested in IDEs or spreadsheets, but when we think about what we want to offer as a whole, I can't help but think of society in a whole different way. The further away you push it, the clearer it is that it is not about adding specific features, but about listening to what users need and explaining how we can solve the problems they see; to be there when they need assistance and to help them be secure rather than adventurous.

We created KDE Espa\~{n}a to be able to sustain Akademy-es\todo{add footnote explaining Akademy-es} initially, but over time it has evolved into a platform with a much broader communication spectrum including Akademy-es, a magnificent blog, podcasts, trainings and conferences. This is also important because as soon as you start communicating, people come to you when they feel unsure about how to help. Furthermore, it helps us stay organized and alive.

Interestingly, one of the initiatives we have started from within KDE Espa\~{n}a has been achieved by narrowing the geographical scope even further. We created a group called Barcelona Free Software, where we offer content from local free software experts and enthusiasts including but not limited to us. This allows people with different sensibilities\todo{find better word} to those who would attend Akademy or Akademy-es, to come and talk to us. The feedback we get at such gatherings is interesting as it shows the kind of problems that our people ache from and how they ask for solutions.

In the end, what this experience has reminded me of is that I am here to make sure society can make the best use of the available technology. Furthermore, in a society increasingly based on information, we need to let it empower the user, the people. We need to remember what the advantages of offering free software solutions are beyond technologies, by remembering that the end goal is that people have the tools they need to create better content in every aspect of their lives, and that includes the possibility to adapt the tools they use.

I think one of the most important motivational factors is to see people being able to adopt the solutions you come up with, and another even more motivational factor is when they use them to create new things. For many of us, it is beyond the comfort zone or even the ideal plan for a weekend but it is worth it to talk to people, it is worth going to that local convention to talk about why and how important it is to grow our society on free software and then gather reasons why that is not the case yet. The response never ceases to amaze me.
