\chapterwithauthor{Baltasar Ortega}{A user in the court of KDE developers}

\authorbio{Teacher by profession, Baltasar Ortega is a user who knows nothing about programming and who loves Free Software and the KDE community. When he's not teaching in his school, he writes a popular blog about the KDE community and its work (kdeblog.com). He is the secretary of and responsible for communication in KDE Espa\~{n}a.}

\noindent{}This is the personal story of how a simple user has become a member of the KDE community and the daily struggle for the success of a project you are passionate about.

This story is written to commemorate the 20 years of the KDE project which I consider extremely important for the development of the software worlwide and in which I feel completely integrated. Sometimes I do not completely understand the language of the developers with whom I communicate almost on a daily basis, which sometimes makes me feel like an astronaut in the court of King Arthur.

It all started when, after many unsuccessful attempts and thanks to the help of thefree software communities forums, I managed to configure a USB router to work with my newly installed GNU/Linux distribution. I think it was an openSUSE 10 and came with the KDE desktop. I simply gave my PC software that I thought better, it started from selfishness.

At that time I did not completely grasp the workings of the ecosystem and its applications but I quickly realized that the KDE project gave me what I wanted: a nice configurable desktop, free from malware, 100\% translated into my language and with convenient applications like Konqueror, Kontact, Kate, Amarok, Digikam, etc.

However KDE Software seemed just something that was "created out of nothing" and I used it without paying much attention about its origin.

Gradually I started to fall in love with KDE and eventually I realized that everyone should enjoy their computer as much as I was enjoying mine.
Thus I was determined to help spreading the word about GNU/Linux and the KDE project but did not know how. I did not know how to code, I could not draw, I did not like to translate and did not know how to package software.

I, as a teacher, knew only how to explain things... so I created a blog about KDE: kdeblog.com, where I would try to help others get started in this wonderful world of ours.

Thanks to the blog I started learning a lot about the world of Free Software; making many mistakes along the way but always learning.
This is how discovered that events were held regularly and discovered that there was a regional group for KDE: KDE Espa\~na... then I applied to become a member... then I was accepted, surprisingly to me... then I decided to attend the annual meeting: Akademy-es 2010 in Bilbao. That meant traveling 600 km from where I live to an event where I knew no one except for a few email exchanges. It was the experience that definitely changed my view of Free Software.

They welcomed me as one of them and I discovered what KDE really is about. There I discovered that behind each application, each translation, each design on my computer there is a person who made it possible. There I discovered that these people have great ethical and moral values and there I discovered I wanted to be a part of this gear.

Suddenly it was no longer only KDE Software; KDE was a project of people making software for other people, respecting their freedoms and privacy. That was a great discovery for me, I wanted to be an active part of the KDE community and help spread it.

I understood that the KDE community consists of all kinds of people, not just coders as one could have assumed, and that myself as a user without any coding skills, could play an important role in the development of Free Software.

The rest, as they say, is history.

In that year I became "a user in the court of KDE Developers" working every day to promote Free Software from my particular point of view, either on the blog, in social networks, by giving talks, putting stickers on my computer, telling my students about the free alternatives, collaborating with projects such as Wikipedia or OpenStreetMaps, installing GNU/Linux on my friends' computers, organizing local events such as the 15th anniversary of KDE party or the Jornadas Libres de Vilareal - all without knowing how to write a single line of code.

A thank you comment, being able to travel, learning languages, learning new working methods and understanding that I am part of a community that works for the good of humanity, all this has given me great pleasure. But above all it has given me the opportunity to meet extraordinary people who devote so much time to creating a better world.

Finally, I want to thank all the people who have contributed, contribute and will contribute with all their work and their effort to keep the KDE community active and thriving. It will always give you back more than you contribute.
