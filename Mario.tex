\chapterwithauthor{Mario Fux}{Remote Places Make Magic Possible}

\authorbio{Mario Fux is the father of two sons, husband, teacher, webmaster and a reader of almost anything. He is using Free Software for nearly 20 years and joined KDE around 2007. He is the main organizer of the annual Randa meetings - KDE's week-long tech summit - and helps where help is needed.}

\noindent{}KDE was born twenty years ago, in the last millennium. Back then, I didn't know about it and its great and welcoming community. Back then I made my first experiences with GNU, Linux and distributions and thus Free Software in general. I think it was my first distribution, SuSE 5.3, that introduced me to KDE. KDE 0.9 or 1.X, I can't remember because I didn't stick to Linux at first but went back to Windows or maybe dual-booted. But something made me stick to the Free Software world and when people asked me about why I spent so much time on something I didn't earn any money from, and would spend hours or even days siting in front of a PC, I told them: “It's because everybody told me that you don't even get kicked for free these days, but here I found something where people are working on something in their spare time, because they love it with passion and this something even challenges multi-billion IT-businesses en passant.”

So the Free Software movement became a big hobby of mine or better say, a big part of my life. I visited several Linuxtage in Germany, other Linux events abroad and then in 2007 Anne-Marie Mahfouf invited me to Akademy in Glasgow, Scotland. She is of course famous for her work in KDE Edu but we knew each other from a monthly column I wrote about Free Software in education: TUX\&GNU@school. Around this time I was really active in the area of Free Software in education as I was a trained primary school teacher. I gave a presentation about my local school that I migrated to GNU/Linux and KDE software including old Windows education software under Wine. It was a very interesting week and although I was (and probably still am) quite shy I made some new contacts.

Back in Switzerland where I grew up and still live, my first Akademy experience began to affect me. I began to use more and more KDE software, and started to study computer sciences as my minor. I realized that I love to write code but it is not my biggest strength or talent. So how could I give back something to this community that offered me software I use daily for free and with pleasure? In the Free Software media I read about KDE sprints that happened frequently and there it was: the idea to host such a sprint. It was around the time that Aaron Seigo, one of the central people in KDE at the time, was in Zurich for a presentation. I studied there and thus could visit his presentation and used the opportunity to tell him about my idea and that I would invite him and the other Plasma hackers to come to Randa where I would offer them a place to sleep, electricity, some internet connectivity and a beautiful and distraction-free surroundings, and inspiring nature.

In August 2009 it really happened: more than 15 women and men hackers from around the world arrived in a village with a population of less than 400 people in the middle of nowhere. They all arrived safely at my parent's Chalet (local holiday house) and although I didn't know them the Plasma team felt like family immediately. For a week they hacked, discussed, hiked and wrote software that I and millions of other people use and even better: can and will use in the future.

It was an amazing week for everybody and it just couldn't stop there. We needed to find a bigger house for more people. And there is a bigger house in town; actually the biggest house in Randa that we rent since 2010 for one week a year. That's the story of how the Randa Meetings were born.

(Almost) every year since 2009 several dozen Free Software enthusiasts come to Randa and spend a week eating, sleeping, discussing, hacking, deciding and contributing under one roof. This wouldn't be possible without the help and support of a lot of other people: my wife, children, parents, brother and even my aunts and uncles are helping. And there are a lot of Free Software people not part of my family that help as well. It's a lot of work, a lot of joy and always great to see what can be achieved in one week with focus.

Over the years we saw important decisions made in Randa like at the Platform 11 discussion which led to the successful KDE Frameworks 5 releases. The KDE Multimedia team prepared some great Amarok releases in Randa and discussed with Qt the future of Phonon. Kdenlive, one of the best video editors in the world, not just in Free software, had some important meetings in the middle of the Swiss Alps and decided to formally become part of the KDE community. The KDE Education group worked tirelessly in Randa days and long nights and now includes GCompris, which has been ported from Gtk to Qt, and has become part of the KDE community. And in recent years we worked hard to bring KDE software to new devices and even proprietary platforms as I strongly believe that even users of proprietary operating systems should be able to use our great software and thus know about us and be able to support us.

Although I strongly believe in the values of a Free operating system and will most probably always use one of them on my main systems I think that we should bring our values and software to other systems as well. And we should not be afraid of asking for money, either. The most important software distribution channels now are the Google Play Store and the Apple Store and I'm convinced that KDE should be there. Actually we are already there - some of ours apps are included and downloadable for millions of users. But let's ask them for a Franken, Dollar or Euro or two and we'll get the money and can use it to support development.

Back to the farther future of KDE. I'm sure we will still be active and healthy in five, ten and then twenty years and I hope that the Randa Meetings will be there too. It would be great if at least one of my kids would still be involved in these great meet-ups in the middle of Europe. And if I had a wish for KDE I'd say KDE should become more self-confident; we have such great software and values. More people should know and use our software, and embrace our values. If we stay such a welcoming community and continue to open up in areas like promotion, artwork, design and communication, we'll become even more successful. As we become even more international, more people on this planet will know about our great projects and use our software and thus become part of this great community.

About the first train that drove through the Matter Valley, where Randa is located,125 years ago, it was said: “Great people create great things, but good people create the perpetual.” KDE people are good people, creating for the future.
