\chapterwithauthor{Mario Fux}{Remote places make magic possible}

\authorbio{Mario Fux is the father of two sons, husband, teacher, webmaster and a reader of almost anything. He is using Free Software for nearly 20 years and joined KDE around 2007. He is the main organizer of the annual Randa meetings - KDE's week-long tech summit - and helps where help is needed.}

\noindent{}Twenty years ago, in the last millennium, when KDE was born I didn't know about it and its great and welcoming community. Back then I made my first experiences with GNU, Linux and distributions and thus Free Software in general. I think it was my first distribution (SuSE 5.3) that introduced me to KDE. KDE 0.9 or 1.X. I can't remember and I didn't stick to Linux at first but went back to Windows or at least dual-boot. But something made me stick to the Free Software world and when people asked me about why I spend so much time on something I didn't earn any money from and sit in front of a PC for hours and days I told them:
``It's because everybody told me that you don't even get kicked for free these days but here I found something where people are working on something in their spare time, because they love it and with passion and this something even challenges multi-billion IT-businesses en passant.''

So the Free Software movement became a big hobby of mine or better a big part of my live. I visited several Linuxtage in Germany, other Linux events abroad and then in 2007 Anne-Marie Mahfouf invited me to Akademy in Glasgow, Scotland. She is of course famous for her work in KDE Edu but we knew each other from a monthly column I wrote about Free Software in education: TUX\&GNU@school. Around this time I was really active in the area of Free Software in education, which made quite a lot of sense as I was a trained primary school teachers and thus I gave a presentation about my local school that I migrated to GNU/Linux and KDE software including old Windows education software under Wine. It was a very interesting week and although I was (and probably still am) quite shy I made some new contacts.

Back in Switzerland where I grew up and still live my first Akademy experience evolved its effects.\todo{native speaker check} I used more and more KDE software, started to study computer sciences as my minor but realized that I love to write code but it is not my biggest strength or talent. So how could I give back something to this community that offered me software I use daily for free and with pleasure? In the Free Software media I read about KDE sprints that happened every other week and there it was: the idea to host such a sprint. It was around the time that Aaron Seigo, one of the central people in KDE at the time, was in Zurich for a presentation and as I studied there and thus could visit his presentation I used the opportunity to tell him about my idea and that I would invite him and the other Plasma hackers to come to Randa where I would offer them a place to sleep, electricity, some internet connectivity and a beautiful and distraction-free surrounding and inspiring nature.

And then in August 2009 it really happened: more than 15 female and male hackers from around the world arrived in a village with a population of less than 400 people in the middle of nowhere. But they all arrived safely at the Chalet (local holiday houses) of my parents and although I didn't know them it felt like family immediately. For a week they hacked, discussed, hiked and wrote software that I and millions of other people use and even more can and will use in the future. It was an amazing week - for everybody and it just couldn't stop there. We just needed to find a bigger house for more people. And there was a bigger house in town - actually the biggest house in Randa that we rent since 2010 for one week a year. That's the story how the Randa Meetings were born.

(Almost) Every year since 2009 several dozens of\todo{native speaker check} Free Software enthusiasts come to Randa and spend a week eating, sleeping, discussing, hacking, deciding and contributing under one and the same roof. It's actually quite similar to a hotel I lead and organize for one week a year. And this wouldn't be possible without the help and support of a lot of other people: my wife, my kids, my parents, my brother and even my aunts and uncles are helping me. And there are a lot of Free Software people not part of my family that help as well. It's a lot of work, a lot of joy and always great to see what can be achieved in one week with focus.

Over the years we saw important decision made in Randa like at the Platform 11 discussion which led to the successful KDE Frameworks 5 releases. The KDE Multimedia area prepared some great Amarok releases in Randa and discussed with Qt about the future of Phonon. Kdenlive - one of the best (and not just Free Software) video editors had some important meetings in the middle of the Swiss Alps and the KDE Education group worked tirelessly in Randa during long nights. And in recent years we worked hard to bring KDE software to new devices and new (or rather old ;-) platforms as I strongly believe that even users of proprietary operating systems should be able to use our great software and thus know about us and be able to support us.

And although I strongly believe in the values of a Free operating system and will most probably always use one of them on my main systems I very much believe that we should bring our values and software to other systems as well. And I don't think we should be afraid of asking for money on some of them. The currently most important software distribution channels are the Google Play Store and the Apple Store and I'm convinced that KDE should be there. Ok, actually we are already there and some of ours apps are there - downloadable for millions of users. But lets ask them for a Franken, Dollar or Euro or two. We'll get them.

But back to the farther future of KDE. I'm sure we'll still be there in five, ten and then twenty years and I hope that the Randa Meetings will be there too. It would be great if at least one of my kids would be still (yes, still ;-) involved in these great meet-ups in the middle of Europe. And if I had a wish for KDE I'd say KDE should become more self-confident: we have such great software and values that more people should know about them and use our software. If we stay such a welcoming community and be a bit more open in not too technical areas like promotion, artwork, communication and co we'll become even more successful. And then we become even more international and more people on this planet (and other planets and galaxies) will know about our great projects and software and use it and thus become part of this great community.

There is this old quote regarding the first train that drove through the Matter Valley (where Randa is located) 125 years ago: "Great people create great things but good people create the perpetual."