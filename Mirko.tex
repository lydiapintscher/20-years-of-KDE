\chapterwithauthor{Mirko Boehm}{KDE e.V. - the Backbone of the KDE Community}

\authorbio{Mirko Boehm became a KDE contributor in 1997, where he
  worked on libraries and applications. He was elected a board member
  of KDE e.V. in 1999, and served in all board capacities from
  treasurer to president until 2006. He is the author of ThreadWeaver,
  the concurrency scheduler framework in KDE Frameworks 5. Until 2016,
  he contributed to KDE as a coder, an auditor and member of the financial
  working group. Mirko Boehm is a director at the Open Invention
  Network and the CEO of Endocode. He teaches open source economics at the Technical
  University of Berlin. Mirko lives with his wife and two kids in
  Berlin.}  

\noindent{}In 1996, I was a university student who learned programming
in C++ using an incredibly expensive Unix work station. GNU/Linux was the
obvious path to program on a similar system after school. However, it
was the time of FVWM - a graphical user environment that mainly
consisted of a rudimentary window manager with little visual charm
and no usability. Using GNU/Linux felt like flying a jet plane, it was
very powerful to the 
initiated, as long as a the user knew what all the knobs and buttons
did and all the secret tricks to keep it going. When I learned about
KDE, a free software project to build a 
user-friendly and powerful graphical desktop, I was hooked. I
started coding on KDE, learned to love C++ and Qt and grew to be a
part of the community. My motivation was mainly to contribute
code. There where about two dozen KDE developers at the 
time, and the community grew quickly. In 1998 I went to LinuxTag in
Kaiserslautern. I met many of the other contributors for the first
time there. We camped in Kaiserslautern university lecture halls, in sleeping bags
on the floor. We talked about
visual features and interaction paradigms, performance, communication
protocols and software freedom. When I was asked if I wanted to be a
member of KDE e.V., I was a bit surprised. Why would a free software
community need a registered organization to represent itself?

KDE e.V. is a charitable organization that represents KDE
in legal and financial matters. As a legal entity, it is able to 
accept donations on behalf on the community, and to own the trademarks
on the KDE name and logo. I was convinced and I joined. The ``e.V.''
in KDE e.V. stands for ``eingetragener Verein'', or registered
association, in German. This concept exists in Germany in this form
since the year 1900. Essentially, it is modeled after a shareholder
company where each member (shareholder) holds the exact same amount of
shares. Choosing this legal structure implies an important trait of the
KDE Community - it is driven by individual contributors, and those
contributors are considered equals among peers. If the organization
owns assets like a trademark, this asset is effectively jointly owned
by the members of the organization. The concept of ``eingetragener
Verein'' implements the ideals of volunteer driven free software
communities very well, even though it was contrived long before
software was a thing.

In 1999 I was elected to the KDE e.V. board of directors. There where
two main tasks - to attain charitable status for KDE
e.V., and to rewrite the bylaws so that they are supportive of
the large free software community KDE evolved into.

Charitable status for KDE e.V. was not yet granted. Being a charitable
organization in Germany is not the same as being a not-for-profit. The
organization has to pursue a charitable goal. To a
casual observer, it seems obvious that facilitating the development of
free software contributes to the common good. Not so for the German
financial authorities, who fail to find this aim in the finite
list of potentially charitable activities, which does include the
advancement of marriage and family or of religion. KDE e.V. finally
managed to be granted charitable status, if only through twisting its
purpose to fit into the predefined schemes. Once that was achieved,
KDE e.V.'s formal structure -- a charitable organization with the goal
to advance free software and that considered all its members as
equals -- fits the ideals of a volunteer-based free software community
like a glove. It seems like the
German legislators would be well-advised to add the advancement of
free software to the list of blessed charitable activities.

Re-writing the bylaws proved to be a challenge, because it meant to
lay the foundations of governance of the KDE Community as it continued
to grow and began to show signs of needing formal structure.
To represent the community that understands itself as made up of
individual volunteers to the product KDE, the rule that only natural
persons could be ``active'' members (members with a vote) in the
organization was kept. To protect the self-determined administration of
the community, a rule was adopted that to become a member, a
contributor had to be invited by an existing member and endorsed by
two more. Considering that every member became a co-owner of all
KDE-related assets, such protection was necessary. Directed at
preventing hostile takeovers of the organization 
(the community was still not very big in numbers), this invite-only
rule later contributed to the image of KDE e.V. as an exclusive
club. Organizations could become ``supporting members'' if it was in
their interest to advance KDE, however this form of membership did not
grant them a vote.

Companies that became a part of the community
gained a say indirectly by having employees that worked on KDE invited
to be members. This was an intended effect that preserved the focus on
the individual contributor while still involving
organizations. Underlying that was an implicit understanding that the
KDE Community should be innovative and directed by the interests and
motivation of the contributors, as opposed to commercial interests
regarding, for example, the stability and maintenance of existing
products. On one hand, this enabled the KDE Community to drastically
innovate over multiple generations of its desktop. On the other hand,
the rule shut out commercial contributions into the stability and
maintenance of existing product versions. As a result, KDE became a
beautifully designed and technologically very advanced desktop that
sometimes crashed. The focus on individual contributions is based on
the idea that self-identification, choosing what to work on based on
individual skills and motivations, is the best approach for a software
community to stay ahead of the curve. It has served KDE well and
became the yardstick for community management when free software
finally became mainstream in about 2010.

Similarly important to what an organization is supposed to do is what
it is {\em not} supposed to do. KDE e.V. is not supposed to steer or
influence the technical direction of the development of KDE. Its
purpose is to be a support organization for the community. In a
community that primarily produces software, it is however hard for
anyone who participates not to influence technical direction. In the
case of KDE e.V., it organizes the annual KDE conference, and finances
sprints and other activities. The choice of which development sprints
to sponsor or where to be present at a conference does put the
spotlight on specific technologies and may influence technical
direction. 

An important goal of the new bylaws was to account for the regular
fluctuation of contributors and to retain the experience of those who
for personal or other reasons stopped contributing. The new bylaws
introduced the concept of ``extraordinary members'', individuals that
are still members of the organization, but waived their voting
rights. It is a way of appreciating their past contributions and
offering the opportunity to be part of debates and conversations,
while concentrating voting rights in the hands of those that are
actively participating.

These ``design principles'' served KDE e.V. well. It grew to 150
members and more, and was able to regularly collect enough donations
to open a permanent office and hire an administrator. It regularly
hosts the annual Akademy conferences and runs the hardware and
software of the KDE Community. It holds trademarks and represented the
community in legal matters. It has become an example for how a free
software community can handle its affairs in a self-determined 
way. Today, KDE e.V. is a large organization with multiple working
groups and stable, established processes.

Minor modifications aside, the current setup of KDE e.V. is more or
less unchanged since 2003. Over time, some areas have surfaced where
reform may be necessary, or is already underway. Many of the
discussions between the members 
have been conducted on the KDE e.V. membership mailing list, which is
not open to the public or all contributors. Having such a private list
was deemed necessary, but it excluded some contributors from taking
part in these discussions. After heated debates at Akademy 2012, a
public community mailing list has been created. The norm is now that all
debates should be held there unless they require discretion. By
preferring lazy consensus and prolonged debate, controversial decisions
are hard to make, leading to a practice of bike-shedding.
Even though an online voting mechanism has been implemented, it was
rarely ever used to ensure that debates could be concluded eventually,
and with a productive outcome. This gave vocal minorities or even
individuals a de-facto veto and inhibited the implementation of
majority opinions. A decision making culture based on consensus works
wonderfully well for small groups, but stifles large organizations.
Since KDE e.V. does not provide technical leadership to
the community, the board and the organization are limited in what they
can achieve or in providing direction. A balance needs to be found
between maintaining the freedom of the community to innovate, and the
ability to coordinate activities within a large group. And finally,
there is a lag between changes in the composition of the overall
community and the time those are reflected in the membership
structure. This provides some stability, but also disconnects the KDE
e.V. membership from the larger KDE Community. It is paramount
for KDE e.V. to perform well that its membership overlaps as much as
possible with the active core contributors of the community.

Most important of all, KDE has changed from being a project that is
all about software to a community that is all about people. The KDE
Community now invites other projects to come under its umbrella. This
marks an important cultural and technological shift, and raises new
questions. Should the new sub-projects be represented in KDE e.V., and
how? Since the community still consists mainly of volunteers, the
design principles of the KDE e.V. bylaws still apply. The basic
principles are sound. However, KDE e.V. needs to evolve its processes
and rules at the same pace as the community changes and re-invents
itself. It will also need to review if the ideals embedded into the
current organizational setup still represent why contributors
participate in the community. 

I am glad that I was offered to opportunity to join KDE e.V. in 1998
and later had the opportunity to serve on the board. After 20 years,
KDE e.V. is the backbone of the KDE Community, it provides structure
process and continuity. With a bit of maintenance, careful adjustments
and by putting the community and its ideals at the center, I am sure
the next 20 years can be just as successful.
