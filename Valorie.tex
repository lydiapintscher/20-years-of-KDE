\chapterwithauthor{Valorie Zimmerman}{Why I Chose KDE, and Why KDE Is Family}

\authorbio{Valorie Zimmerman has been using KDE software since the KDE 3 days. After meeting some KDE women in Linuxchix, she became active in Amarok documentation, then the Community Working Group, and also on the distribution level, the Kubuntu community. Valorie also helped build the Student Programs team, participating in the first Google Code-in for pre-college students, administering Season of KDE and Google Summer of Code. Building community is what it's all about.}

\noindent{}Around 2001, my son showed me two desktops, and asked me to choose which one I wanted on my new install of Mandrake. I chose blue, the pretty one, and so began my KDE journey. Back then, I understood very little besides that I found KDE software to be not just more beautiful, but also usable and configurable. 

Before contributing to KDE, life was busy with kids and work. Fortunately a friend introduced me to Linuxchix, so a friendly group who helped out newcomers welcomed me in. I met people who contributed all up and down the stack, including Windows users like me who were considering a move to Linux. So Linuxchix was my way in to KDE and Kubuntu.

Back in those KDE 3 days, KDE was still a bit controversial in free software circles. Qt licensing concerned some people and the KDE vs. GNOME rivalry seemed real and even upsetting to some advocates of free, libre software.

If any group wants to grow, it must have ways in. For KDE, I found IRC and mailing lists first. Many women on the internet find the hostility and bad behavior discouraging enough to never find their way in. Fortunately I had the Linuxchix at my back, and found KDE mostly welcoming. When I spoke up to volunteer, people enthusiastically replied, and gave me good suggestions about how to start. I found that every project needs better documentation! Growing groups will do everything they can to help you get started. I began in Amarok, which was such fun. After looking through old docs, and considering the options, we decided to do the Amarok Handbook on Userbase, KDE's user wiki. Not realizing that this was somewhat pioneering, we just plowed ahead, and used the old docs to make new pages in Userbase. It was great to work with Amarok developers, KDE documentation folks, along with the wonderful web team that has been keeping the wikis so useful to us. 

A word on volunteering. Early in my "KDE career" I volunteered for the Community Working Group. Little did I know what I was getting into! That said, tackling the difficult issues between developers has been rewarding, because peace is usually the result. At the very worst, we know we tried our best to create a good situation out of a bad one. So blithely heading towards the cliff-edge like The Fool on the Tarot card, has been a good move. Getting to know people deeply is wonderful. Being a part of the Community Working Group has continually reminded me why KDE is family. Learning how to listen better, how to "fight fair", how to recognize issues as they are forming, and then helping them go in good directions, have made all of us involved better humans. I salute all the past, present and future members of the team, and peace-makers all around the world.

When the first Google Code-In was announced, those of us on the Amarok Handbook project immediately thought about all those unwritten pages, and began creating tasks for students; one task equaling one page. What a fantastic experience! It was wonderful to interact with those kids (ages 12-18) and help them help us. Once involved in Google Code-in, which was intense, exciting and productive, I was hooked on student programs in KDE. Looking back at the students who have worked with us over the years, one can see that student programs create the future of KDE. Our students begin working with mentors and teams, and our goal is always to welcome them into the family of KDE. Many of our most successful initiatives are now being run by former students, and many of our most effective mentors were once students themselves. Working with the Student Programs team has been so satisfying.

One of the things I love best about KDE is that while we do the inward-looking work, such as improving our processes, governance, and social relationships, we are generally outward looking. We don't simply make software that pleases us, but follow our ideals when working. I see developers write tests for their software even while complaining about how boring it is, simply because they value quality. I see people step up to write stories for KDE's official news site, release announcements and blog posts even when they dislike writing, because they want to share our work with the world. People comb the code for spelling errors and other small issues, and quietly fix them. Document writers do the same thing with the documentation -- fix errors, test the texts to be sure they are up-to-date, contribute new screenshots -- all quietly and often un-acknowledged. Our Visual Design Group created itself out of nothing, and quietly steps in to help application teams, Plasma developers, and web-workers. The Sysadmin team is tireless! They keep our infrastructure up, humming along happily, and most important: securely. They create new structures to help out the developers, and improve old ones. We have folks working upstream in Qt, making, for instance, accessibility Just Work for everyone. Our distributions are richer and healthier with KDE involvement, along with many other FOSS projects in the ecosystem that surrounds and supports us in turn.

All of this work affects all of us in KDE every day, but we may not notice it unless there is a problem. The focus on freedom and quality is now even moving onto platforms beyond Linux and BSD in a major way. Even folks who don't use Android, Mac or Windows have generously contributed to support developers making our applications usable to an even wider population. This generous spirit and welcoming attitude is what has kept me, a grandma, involved in creating KDE as long as I am able.