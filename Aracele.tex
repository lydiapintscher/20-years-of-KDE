\chapterwithauthor{Aracele Torres}{KDE ceased to be software and has became a culture} 

\authorbio{Aracele Torres is \todo{add bio}}

\noindent{}KDE as a software project was born in 1996 when German programmer Matthias Ettrich realized that Unix-based systems were growing, but their interfaces were not user-friendly enough for the end user. It was only three years after the first GNU/Linux distributions had begun to appear and Matthias noticed the absence of a graphical user interface that offered a complete environment for the end user to perform their daily tasks. He thought that for the people who began to consider GNU/Linux as an alternative to proprietary systems, a beautiful and easy to use graphical environment would help a lot. Thus was born the project "Kool Desktop Environment" or simply "KDE". The name was a pun on the proprietary graphic environment very popular at the time, CDE (Commond Desktop Environment) that also ran on Unix systems.

One year after Matthias Ettrich's announcement inviting developers to join the project, the first beta of KDE was released. Nine months later, came the first stable release. The dream of Matthias and his community of contributors was becoming a reality and occupying an important place in the history of free software. The project was maturing and becoming more complex and more complete. Thanks to the collaboration of people worldwide, KDE had grown from version 1 to 2 in 2000, and then to 3 in 2002. In 2008, after very important changes, the community launched the revolutionary version 4. In 2014, the equally innovative version 5 came out that showed in its visual design, framework, and applications that the community is ready for the future.

During that time a lot has changed in the community in addition to its technologies. There was a change in its name and its identity. In 2008, the community began to refer to "KDE" no more like a software project, but as a "global community." This identity change was made official in 2009, when the community announced its rebranding. At the time, it was announced that the name "K Desktop Environment" would be abandoned because it no longer represented what the KDE had become. This long name had become obsolete and ambiguous, since it represented the desktop that the community has developed. At that time "KDE" was not just a desktop, it was something bigger, as the announcement said: “KDE is no longer software created by people, but people who create software”. 

Many may not have noticed, but this rebranding was a turning point in the history of KDE. In it the community makes clear its tendency to perceive and keep up with state of the art of computing. The reign of desktops was over and it made no sense to limit KDE. Therefore, the decision to use only the name "KDE" intended to communicate to users the message that the community was attentive to the future. "KDE" would no longer be synonymous with a limited set of software components, but an international community that produces free technologies for the end user, whether for desktop or mobile devices or other technologies that are yet to come. 

In 2012, this process of change was synthesized in a manifesto. The "KDE Manifesto" listed the core values advocated by the community and it worked also as a kind of warning\todo{find better word}: the community was open to house new projects under its umbrella brand. In 2014 the KDE Incubator was started. It is an incubator of projects that join the community and have the same benefits as other native projects. This incubator is now home to a variety of projects, from a wiki dedicated to educational topics to a distribution. 

So the community continues its trend of the past 20 years, which is inclusive growth. In the past it has grown to provide its users a better desktop experience. At present it continues expanding and changing also to offer the best experience on mobile devices. In the future it is very likely that it starts to create products for a new generation of devices: cars, smart TVs, refrigerators, stoves, etc. It is possible that smart homes, maybe even an entire city can use the technologies of the KDE community to work. Why not? 

For almost 10 years I have used the technologies that the KDE community produces. I remember I started with KDE 3.5, when I still did not even know about the social importance of free software. As a user, as a contributor and as a historian, to look at these 20 years of history, the feeling I have is the same. The KDE project was born out of the concern\todo{find better word} of a group of people to make computing, especially free and open computing, accessible to all. In the 1990s, when it was created, it was a time when the GNU/Linux systems were becoming popular and the internet was also becoming more present in people's lives, especially after the rise of the web. KDE then emerges as an important tool in the popularization of free software. It was an interface that helped the man\todo{find better word} to understand the machine and communicate through it. Today KDE is an interface between people. It is what unites and connects them towards free computing. It is a community that encourages the growth of people and projects, that seeks innovation, which defends the free sharing of information. KDE ceased to be software and has became a culture.
