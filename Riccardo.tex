\chapterwithauthor{Riccardo Iaconelli}{Software, Freedom and Beyond}

\authorbio{Riccardo Iaconelli is a KDE developer since his youngest age\todo{native speaker check}. He made his first steps in the open source world within FSFE and never stopped. He counts in his portfolio important collaborations with entities such as CERN or INFN. His latest project is WikiToLearn, a web platform dedicated to the creation and exploitation of free scientific knowledge.}

\noindent{}I have been a KDE developer for more than half of my life now. KDE is turning 20 this year, making it one of the longest living (and most successful) open source communities in the world. I have performed almost any kind of tasks in all those years: I have been coding, translating, doing graphical work, promotion, even event organization or paperwork, when it was needed. I am so grateful to the community for the experiences I have made, that I am writing this essay with slightly wet eyes. KDE has been a central part of my education, one I could not do without. KDE is, today, about to face many new challenges. To be able to tackle them effectively, we need to know what made us strong, during all this time. I don't presume to have the answer, but I will share my story, in the hope that it will be useful.

So, how did I get interested in Free Software, in the first place? Well, I had the luxury of learning how to code at a very young age: I was 9 and I wanted to build my own website about skating. So I started to learn what I needed: HTML and JavaScript. And then, of course, some PHP. But I got so intrigued by the programming world that I decided I was going to be a hacker, when I grew up. After much googling, I figured that the only way I could be a hacker, though, was to install Linux. I downloaded my first Linux distribution, a SuSE, and I installed it like I would with any other Windows application: I just kept pressing "Next" without caring to read what was going on. A minute later, I was there, without a bit of all of my data, applications and music, with an operating system I did not know, with no internet connection, but a strong will to "become a hacker" and learn. I installed and used just about every application I had on the SuSE DVD (I even went as far as installing and using an application to print out barcodes for hours). But it was not just technical work. I educated myself about the four freedoms of the GPL and why they mattered. And I found the community I wanted to be a part of: KDE.

I was, however, too shy to contribute to KDE through coding, even if I already knew some C++, so I started with the simplest job I could take on: translating. And I had lots of fun! So I continued, I became a core developer of Plasma, writing the first plasmoids, a core developer and a designer of Oxygen (working on the theme, window decoration, cursor theme, icons, wallpapers...) and many more things (from kdelibs to games to PIM). Probably the single piece of work (outside these big projects) I am most proud of is the complete redesign (and implementation) of Amarok's user interface in QML. It was sexy, but unfortunately it was never released.
By getting my hands dirty I learned a lot from everyone in the community, trying to build new skills and expertise, to be used later. It was simply amazing.

There was not a single role model, but I had the luck of finding many great mentors within the community. In general, I tried to simply be attentive to other people, trying to find out what I like in what they do, and what I can learn from them. Once I see it, I start to apply those elements in my life, whether it's a personal lifestyle choice, a technical decision or a way to relate to others. This "growing together" is, in my opinion, the true spirit of free software.

It is partly due to this belief that I launched WikiToLearn, a KDE project which works with the greatest research centers, universities and experts, to share under free licenses the great amount of knowledge produced there, at Akademy in 2015 in La Coru\~{n}a. We want to create free and collaborative textbooks, accessible to the world and always remixable, to teach the values of openness and collaboration to the next generation. I think that teaching why openness is important is crucial, if we want to keep the open movement as strong as it is now.

To motivate myself I always remind myself why I am doing what I am doing. I believe in a free world; I believe in the power of decentralization. I believe that the sum of our collective minds and efforts is greater than what any of us alone can achieve. And I want to have fun while doing what I am doing.

I feel that one of the challenges we are facing right now is evolution. How do we grow past a world which was entirely desktop-centric, and which now gives the desktop an important but no longer essential role? I think the answer lies in the community. We have to go back to our origins, to the universities where many of us started contributing. We have to see what is now interesting, explain the fun we can have developing in the open world and the importance of keeping a thriving open ecosystem.

We need to explain our strengths and bring them our experience. Here is what I usually suggest people to do: learn by doing, and always get great mentors to review your work. Get your hands dirty and teach yourself to code. Try, fail, try harder and iterate. And don't stop until you become perfect. Aim for elegance. Learn everything you can from the world and your peers, and never stop doing that. The challenge is with yourself, not with anyone else. How much better can you be?

This way, many years later, I found I have been growing with the community, learning many skills along the way. I watched subprojects flourishing and dying, and in this process shaping the meaning and essence of KDE. In all this period serving as core developer I have seen KDE grow and evolve beyond what we thought was our first goal, to face new and unimagined challenges, and expand towards directions we didn't even believe possible.

I have been a KDE developer for half of my life. KDE turns 20 this year, and I am 25. I want to continue to see this amazing community to flourish, and I want to play my part in the first project that I maintain, by adding yet another global success to our portfolio. And I think that with WikiToLearn we have all the right cards in our hands to achieve that.
The KDE Community has been almost a family to me, and I just want to thank every member of it for making all this a reality.
